%The recent financial crisis has revealed that the quantitative tools of portfolio and risk management were poorly understood not only by practitioners but also by the academic world and have failed to work in times where they were needed most. While theoretically founded portfolio strategies generally fail to outperform completely non-theoretical approaches as the equally weighted portfolio in terms of a wide range of common performance measures \citep{demiguel2009}, it is also true that estimation, model and parameter uncertainty can alter the optimal asset allocation and therefore the the optimal trade-off between the return and risk of the investment substantially \citep{avramov2010,brandt2010}.\\
%
\indent Predicting stock returns is a popular exercise for professionals and academics in finance. It is a challenging task because of estimation and model uncertainty, because of a substantial unpredictable component (shocks) in future stock returns and because successful forecasting models may be exploited by other market participants causing trading behavior that destroys the prediction power of the model. Recent empirical findings suggest that the equity premium is predictable when accounting for model and estimation uncertainty through time-varying parameters, stochastic volatility and by using Bayesian predictive regressions \citep{zellner1965,barberis2000,pettenuzzo2016}. While the amount of predictability from variables such as the dividend-earnings ratio is limited in terms of out-of-sample R$^2$, it can translate into substantial portfolio gains for an expected utility investor \citep{johannes2014}.\footnote{See \cite{rapach2013} for a recent overview on return predictability.}\\
%
\indent Most prediction models relate the future stock returns to past observations of asset specific variables such as the dividend yield or to general macroeconomic indicators such as inflation. However, the market stock price is forward-looking as it reflects the expectations of market participants about the future cash flows of the company. It is natural to expect that using forward-looking information should also be beneficial for stock return predictions. A recent example is \cite{metaxoglou2016}, who improve equity premium forecasts by using option prices. In this paper, we use professional analyst's forecasts to improve asset return predictions. We do so by reweighting the predictive return distributions by a method called \textit{entropic (exponential) tilting} to incorporate the information contained in analysts' forecasts, such as target prices. The idea of this method is to modify the predictive density of the asset returns to match certain moment conditions that are formed based on average analysts' forecasts. The advantage of this approach is that we can combine model-based time-series information with external, forward-looking information in a parsimonious way using closed-form solutions.\\
%
\indent Professional security analysts provide market analyses, make earnings forecasts and give investment advise by providing twelve months forward target prices and stock recommendations. Their reports and opinions can have major short- and long-run impact on stock prices \citep[e.g.][]{irvine2003} and provide a powerful way to disseminate financial information to market participants.\footnote{The US Bureaus of Labor statistics report over 275,000 thousand financial analysts jobs in 2014 (see \url{http://www.bls.gov/ooh/business-and-financial/financial-analysts.htm}, checked on 19.08.2015).} And a vast literature exists about the shortcomings of analyst forecasts revealing skewed incentives (conflicts of interest), herding and biases to please clients that may create market inefficiencies.\footnote{See  \cite{ramnath2008} for an in-depth review of the analyst forecast literature.}\\
%
\indent In this study, we do not evaluate the accuracy of the analysts' forecasts, but we use the (dis-)agreement in the analysts' forecast to regularize the predictive return distribution. In particular, we restrict the mean and variance of the predictive distribution to coincide with the mean and the variance of monthly target price implied expected returns, i.e. simple returns between the current spot and the target price. While we find that restricting the variance of the asset returns is beneficial in terms of out-of-sample performance, as it provides a forward looking measure for (un-)certainty in the market, only restricting the mean has no particular forecasting power. Target prices are usually higher than current spot prices and so there is an upward bias in the target price implied expected returns which is not beneficial for the forecast performance.\\
%
\indent Of course we could also include the analyst information simply as another predictor variable in the predictive regressions, but this would add further parameters and more estimation noise to the prediction problem. Using the analyst information instead in a tilting framework, we only change the shape of the predictive distribution by reweighting it and hence do not require the data to formalize the relationship between asset returns and target prices.\\
%
\indent Another econometric contribution of this paper is the use of a large Bayesian vector autoregressive system to formalize the predictive relationship between asset returns and a great number of predictor variables. To overcome the computational burden that arises in a recursive forecasting exercise, we adopt the so-called \textit{forgetting factors} approach of \cite{koop2013} which allows for all the features recently found to be important to find significant return predictability: Time-varying parameters, stochastic volatility, parameter shrinkage as well as dynamic model averaging and variable selection. The idea of forgetting factors is to approximate the conditional error term variance in a Kalman filter type estimation of the time-varying model, reducing the computational burden significantly. While \cite{dangl2012} used a similar forgetting factor approach for return prediction, we will combine this method with a Minnesota-type prior that restricts the parameter matrix to deal with the case of many predictors.\\
%
\indent It is also worth mentioning that market excess returns, such as the S\&P500, might not be the ideal candidate to find predictability, because of their aggregation over many sectors and industries. In this paper instead, we will investigate the predictability of individual assets by looking at a cross-section of Dow Jones index constituents. We will investigate if these returns are predictable using company-specific characteristics like the book-to-market ratio or by using market and economic indicators such as inflation. Out-of-sample studies for cross-sectional returns are limited and a few exceptions among others are \cite{avramov2002} and \cite{rapach2015}. While these studies use factor and industry portfolio returns, we fill the gap and go down to the level of individual equity asset predictability.\\    
%
%\indent Including many predictors in the prediction model increases its dimensionality and requires some sort of regularization. Here, we will rely on two of such  techniques. We will model a Bayesian vector autoregressive system for the predictive relationship between asset returns and the predictor variables. We allow for time-varying coefficients and stochastic volatility using shrinkage priors and model averaging techniques in the fashion of \cite{koop2013}. The idea of this approach is to use so-called \textit{forgetting factors} to approximate the conditional error term variance in a Kalman filter type estimation of the time-varying model, reducing the computational burden significantly. While \cite{dangl2012} used a similar forgetting factor approach for return prediction, we will combine this method with a Minnesota-type prior that restricts the parameter matrix to deal with the case of many predictors.\\
%
%\indent In a second step, we will reweight the predictive return distributions through a method called \textit{entropic (exponential) tilting} to incorporate external knowledge from analyst recommendations. The idea of this method is to modify the predictive density of the asset returns to match certain moment conditions that are formed based on average analysts' forecasts, for example target prices. The advantage of this approach is that we can combine model-based time-series information with external, forward-looking information in a parsimonious way using closed-form solutions. We will restrict the mean and variance of the predictive distribution to coincide with the mean and the variance of monthly target price implied expected returns, i.e. simple returns between the spot and the target price at each point $t$ divided by twelve. While we find that restricting the variance of the asset returns is beneficial in terms of out-of-sample performance, as it provides a forward looking measure for (un-)certainty in the market, only restricting the mean has no particular forecasting power. Since target prices are usually higher than current spot prices, there is an upward bias in the target price implied expected returns which reduces forecast performance.\\ 
%
 %This is very useful for high-dimensional systems. 
%
%\indent The paper is organized as follows. Section \ref{sec:literature} first summarizes the literature about the usefulness of analysts' forecasts and further motivates the use of traget prices as  and then provides an overview on the state of the art of return predictability for aggregated market indices as well as for cross-sections. Section \ref{sec:methodology} introduces the concept of entropic tilting and describes the applied Bayesian panel vector autoregression setting. Section \ref{sec:application} summarizes the set-up of the empirical study and presents the results. Section \ref{sec:conclusions} concludes and gives an outlook on further generalizations.
%
%\subsection{Sell-side financial information}
%Professional security analysts provide market analyses, make earnings forecasts and give investment advise by providing target prices and stock recommendations. Their reports and opinions can have major short- and long-run impact on stock prices \citep[e.g.][]{irvine2003} and provide a powerful way to disseminate financial information to market participants.\footnote{The US Bureaus of Labor statistics report over 275,000 thousand financial analysts jobs in 2014 (see \url{http://www.bls.gov/ooh/business-and-financial/financial-analysts.htm}, checked on 19.08.2015).} There exists a huge literature about the shortcomings of analyst forecasts revealing skewed incentives (conflicts of interest), herding and biases to please clients that may create %more asymmetric information 
%market inefficiencies.\footnote{See  \cite{ramnath2008} for an in-depth review of the analyst forecast literature.} A  great amount of the research either focuses on the accuracy of accounting related figures such as earnings forecasts %\citep{bradshaw2012a} 
%or on the impact of their publication on market prices to analyze the behavior and expectations of market participants.\\ 
%%A prominent example is \cite{bradshaw2012a} who investigate the predictive accuracy of analyst earnings forecasts against a random walk model and find that analysts' provide more precise forecast only for short horizons and for relative large firms. %For small and young firms and for longer horizons the random walk model dominates. 
%%Studies evaluating the investment value of analysts forecasts often only consider strategies that are based on sentiments on these forecasts, i.e. portfolios having long positions in stocks with favorable forecasts and short positions in stocks with unfavorables ones.\\
%%
%\indent In this study, we will use analysts' recommendations and target prices as predictors for the asset returns and we will use them to provide a signal (moment condition) about the predictive return distribution. %After that, we will form portfolios for a utility maximizing investor using the tilted joint predictive distribution of the asset returns. 
%Analyst recommendations are discrete signals (e.g., strong buy, buy, hold, sell, strong sell) about the current market value of a stock and target prices forecast the stock price over the next twelve months. The two seem to be, besides of various shortcomings mentioned for example by \cite[][p. 1935]{brav2003}% and contrary to earning forecasts that are typically made for a limited period (e.g., a fiscal quarter) only
%, relevant candidates to provide useful investment signals regarding the current and future stock valuation. If this is not the case, does, after all, at least the timing matters, e.g. can recommendation revisions or expected target price returns reveal useful information about future stock returns?
%\indent Already a great number of studies document the investment value of sell-side analyst research: \cite{barber2001,green2006} report portfolio benefits and abnormal returns from simple trading strategies that go long in stocks with favorable analyst recommendations and short in stocks unfavorable ratings. \cite{jegadeesh2004} instead finds that predictive power of the level of consensus analyst recommendations various substantially across assets and proposes to use the changes or revisions in consensus analyst recommendations instead. \cite{jegadeesh2006} applies this to build portfolio strategies that are long (short) in stocks with positive (negative) recommendation revisions and that yield significant positive abnormal risk-adjusted returns against the market. \cite{cvitanic2006} further shows that such simple long-short portfolios can be outperformed by more complicated multi-period expected utility maximizing strategies that are estimated using consensus analysts recommendations. This is not the case for \cite{he2013}, who finds no significant portfolio returns from a \cite{black1992} strategy incorporating analyst recommendations for the Australian stock market.\\
%%
%\indent The evidence for return predictability from target prices is also mixed: While \cite{brav2003} document that target prices are generally informative about future stock prices and that there are substantial short-term market reactions in the stock price to target changes, \cite{bonini2010} find little evidence for target price accuracy using different metrics measuring the prediction error between the target price and the current, twelve month forward and in-between spot prices. This is in line with \cite{bradshaw2012b} who find an average target price premium over the spot price of 15 percent and who report that only two thirds of the target prices are met by the stock price at some time during the forecast horizon in their sample. \cite{lin2016} instead considers changes in target prices, but also find no evidence that institutional trading activity alongside the direction of the target price revisions yields higher risk-adjusted out-of-sample returns. Nevertheless, \cite{da2011} finds that aggregating stocks across sectors according to their the twelve month forward target price implied expected return, i.e. simple return between the current and the target price, yields significant risk-adjusted abnormal returns for different long-short portfolio.
% That is the old text.
%\cite{jegadeesh2004} also find predictive power in the level of consensus analyst recommendations, but it varies across stocks with favorable and unfavorable fundamental characteristics. Instead, the authors propose to use changes in consensus analyst recommendations and find that this yields a robust predictor variables with a positive significant impact on future stock returns. Further, \cite{jegadeesh2006} analyze the effect of recommendation revisions on stock returns for the G7 countries. The authors document significant price reaction on the day and the day after a recommendation change that also translate into an up- and downward stock price drifts in the long-run. Portfolio strategies that are long (short) in an equally weighted index of stocks with positive (negative) recommendation changes in the last 1, 3 or 6 months yield significant positive abnormal risk-adjusted returns against the market. The effects are most strongest in the US, where there is the greatest number of recommendations. \cite{brown2013} is another example to show the impact of analysts recommendation by analyzing the herding behavior of mutual funds. They find that downgrades in recommendations has a larger effect yielding a significant contemporaneous price effect followed by a sharp subsequent price reversal. %Such an overreaction to recommendation reversals can lead to price destabilizing with an increasing level of mutual fund ownership of stocks.  
%In a portfolio context, \cite{he2013} find that stocks with favorable (unfavorable) recommendations yield higher (lower) risk-adjusted return than the in the Australian stock market, using a portfolio strategy with daily rebalancing based on a \cite{black1992} approach to incorporate analyst recommendations does not produce out-of-sample risk-adjusted excess returns after transaction costs that are significantly higher than the market return. %to determine long (short) positions in stocks with favorable (unfavorable) recommendations
%In contrast to this, \cite{cvitanic2006} show that a simple long-short portfolios can be outperformed by more complicated expected utility maximizing strategies. In a multi-period dynamic rebalancing portfolio framework, the authors explore an inter-temporal CAPM that allows for abnormal returns (alpha). The authors derive the optimal strategy for an expected power-utility maximizing investor and estimate the securities alpha using average analysts recommendations. While their optimized strategy outperforms simple long-short allocations, the results are stronger for long horizons and for securities with a greater number of recommendations.
%
%\indent The evidence for return predictability from target prices is also mixed: Generally, \cite{brav2003} document that (i) target prices are informative about the future stock prices, both unconditionally and conditional on contemporaneously issued recommendation and earnings forecast revisions, (ii) that there are substantial short-term market reactions in the stock price to target changes and (iii) that on average, target prices are generally higher than current spot prices (about 30 percent). In contrast to this, \cite{bonini2010} find little evidence for target price accuracy using different metrics measuring the prediction error between current and future (twelve months ahead) spot price, and between target price and the maximum/minimum stock price in between. %The authors also find a negative correlation between the prediction accuracy in target prices and the analysts intensity of research and the market momentum indicating a systematical bias in the analysts forecasts. 
%This is in line with \cite{bradshaw2012b} who find only weak forecast performance in target prices and report an average target price premium over the spot price of  15\%. In there sample, only two thirds of the target prices are met by the stock price at some time during the forecast horizon. This happen less often for high volatility stocks, but more often during time periods of firm-specific positive price momentum and overall large positive market returns.  Most recently, \cite{lin2016} considers changes in target prices and analyzes the institutional trading of a large set of US firms. While they document increased institutional trading activity alongside the direction of the target price revisions, they find no statistical significant evidence that this trading yields higher risk-adjusted out-of-sample returns. \cite{da2011} answer this by arguing that aggregating stocks across sectors according to their the twelve month forward target price implied expected return, i.e. simple return between the current and the target price, yields significant positive portfolio effects. They sort S\&P500 stocks in this way within different industry sectors into nine groups and then form equally weighted portfolios of each group across sectors. Taking a long position in the portfolio with the highest expected target returns and a short position in the portfolio with the lowest expected target returns yields significant risk-adjusted monthly abnormal returns of over 200 basis points. %However, sorting all stocks across their expected target return across all industries at once renders the results to be insignificant.
%\begin{itemize}
%\item While testing the accuracy of analysts' earnings forecasts reveals their incentives, it is natural to use analyst recommendations and target prices for portfolio constructing.
%\item The efficient market hypothesis suggest that all information concerning a stock is already refelcted in its price.
%\item Financial researchers and practitioners have long been interested in understanding how the activities of financial analysts affect capital market efficiency.
%\item Equity analysis provides investors with information on the current and future prospects of listed companies. 
%\item \cite{rapach2015} use a mixed frequency time series model to outperform earnings forecasts from analysts despite their timing and information advantage.
%\end{itemize}
%\subsection{A promising example}
%Before heading to the main analysis, a motivating example is needed. 
\indent Figure \ref{fig:ibmpriceplot} serves as an illustrative example. It shows the IBM spot price, the mean twelve months forward target price and the percentage of buy recommendations from all recommendations (buy, sell, hold) of the IBM stock. While the correlation between the spot and the target price is almost perfect (0.9721), spot price and buy recommendations are negatively correlated (-0.6470). In the plot the target price is almost always higher than the current spot price, indicating an upward bias in these forecasts. The only times when the two plots coincide is after price drops when the spot price starts to increase again, i.e. in late 2002, 2007 and 2014. That is, financial analyst might not be able to predict trend changes (regime switches) but they are able to forecast the price direction. This is in line with the more volatile buy recommendations, which only on average indicate the stock price direction.\\% Intuitively, increasing prices  an high price should lead to fewer buy and more sell and hold recommendations.\\
%
\indent While this pattern is similar also for the other 19 stocks considered here, comparing the target price with the observed spot price twelve months ahead gives a mixed picture. Table \ref{tab:rmsfe} provides the root mean squared forecast errors (RMSFE) between the two for the 20 stocks and compares it to the two year historical mean. Only for 11 out of the 20 stocks the target prices were better forecasts than the historical mean, e.g. for the IBM stock.\\
%
\indent Figure \ref{fig:ibmclsdplot} gives an idea about the predictive power of individual predictors against a benchmark intercept only model by plotting three out-of-sample performance measures for the IBM stock. Considering the first panel showing the cumulative sum of squared forecast errors differentials, we see that mostly all predictors fail to outperform the intercept only model, especially in the financial crisis. The only exception is the log return of the mean analyst twelve months forward target prices.  Note that values above zero indicate that a given predictor has better forecast performance than the benchmark model, while negative values suggest the opposite. The target price itself only has some predictive power between 2008 and 2010 (square markers). This might suggest that not the level of the predicted target price matters, but that the timing of the shifts, represented by the target price implied returns, have predictive power. Also, we see that the average analyst recommendations have no particular predictive power for the IBM stock.\\  
%
\indent The paper is now organized as follows. Section \ref{sec:literature} gives an overview about the findings from other authors trying to exploit analysts' forecasts and reviews the state of the art on return predictability. Section \ref{sec:methodology} describes the applied Bayesian vector autoregression model, which relies on the forgetting factor approach by \cite{koop2013} for large systems. We then introduce the concept of entropic tilting and explain how we translate the analyst forecasts into moment conditions for the predictive return distribution. Section \ref{sec:application} summarizes the set-up of the empirical study and presents the results. Section \ref{sec:conclusions} concludes and gives an outlook on further generalizations.






	