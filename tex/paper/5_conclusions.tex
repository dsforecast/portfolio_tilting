\noindent In this paper we demonstrate that financial analyst forecasts can have predictive power for equity assets of the Dow Jones Industrial index using a novel entropic tilting approach to combine time-series forecasts with analysts' information. We find that the extent of predictability varies across assets and that using Bayesian vector autoregressions with time-varying coefficients, stochastic volatility and model averaging and selection among priors improves return predictions across all assets. While tilting the mean of the predictive distribution towards the target price implied expected returns did not improve forecast performance significantly, tilting the variance of the predictive return distribution towards the implied expected target return variance produced forecasts outperforming a simple intercept only model. This may be explained by the fact that the tilted densities are more often stronger concentrated around the true outcome than the baseline density. In other words, the agreement among analysts reduces the predictive variance of the asset returns. Contrary, the disagreement among the analysts may be an indicator for future market uncertainties.\\
%
\indent Using entropic tilting has several advantages: It can incorporate any kind of (forward-looking) information into a predictive regression framework in a parsimonious way without increasing estimation noise. Hence, it is a regularization approach that is suitable in high-dimensional settings, even portfolio problems. %This is very similar to other portfolio regularization techniques as \cite{brandt2009}, who factorize portfolio weights in terms of asset characteristic such as the dividend yield and maximize the average utility of the investor.
Notably, by using the analyst information in a tilting framework we only change the simulated draws of the predictive distribution and hence we do not require the data to formalize and estimate the true relationship between asset returns and analyst target prices. However, while in this way the dimensionality of the forecasting model, in this case of a Bayesian vector autoregression, is unchanged, we do not account for the stochastic nature of the tilting information.\\
%
\indent This opens the door for possible extensions: While one could develop a tilting framework that not only considers the set of predictive distributions that strictly fulfill the moment conditions, one could try to search among all possible predictive distributions and then minimize the distance of the target moments given a statistic or economic criteria. Second, one could also apply the tilting framework to predictive portfolio weight regressions. For example, the framework of \citep{frey2016a} may be used to incorporate external knowledge directly about the portfolio weights, i.e. a certain target allocation, instead of tilting the underlying return process. Finally, the tilting approach is also applicable for panel VAR systems that model various assets simultaneously. Tilting the joint predictive return distributions may then be used in a portfolio allocation problem of an expected utility maximizing investor.
%While one could analyze different assets with other financial characteristics than Dow Jones index companies, econometrically, it would be interesting to see even larger portfolio dimension and to see how entropic tilting can be used as a regularization device itself. Bayesian factor models and model combination schemes might be also possible to overcome the difficult choice of the right predictor variables.

%\subsection*{How to incorporate external knowledge into a portfolio optimization framework?}
%Incorporating non-sample information has a long tradition in portfolio strategies and the Bayesian toolbox offers the great advantage to incorporate subjective information through informative priors. The most prominent example is \cite{black1992}, who form 'posterior' estimators for the mean and variance of asset returns from (i) a market equilibrium model and (ii) formalized investment views through a mixed estimation method. Another example is \cite{brandt2009}, who factorize portfolio weights in terms of asset characteristic such as the dividend yield and maximize the average utility of the investor. Although the primary idea for this approach is to reduce the number of estimable parameters to stabilze the portfolio weights, the characteristics of the stocks also enrich the information set of the investor. \cite{frey2016b} elevate from this in a hierarchical Bayesian regression framework centered on the optimal factor weights and show promising out-of-sample results. 
