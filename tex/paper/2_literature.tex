\subsection{Analysts' forecasts}
\indent A great number of studies document the investment value of sell-side analyst research: \cite{barber2001,green2006} report portfolio benefits and abnormal returns from simple trading strategies that go long in stocks with favorable analyst recommendations and short in stocks with unfavorable ratings. \cite{jegadeesh2004} instead finds that the predictive power of the level of consensus analyst recommendations varies substantially across assets and proposes to use the changes or revisions in consensus analyst recommendations instead. \cite{jegadeesh2006} applies this to build portfolio strategies that are long (short) in stocks with positive (negative) recommendation revisions yielding significant positive abnormal risk-adjusted returns against the market. \cite{cvitanic2006} further shows that such simple long-short portfolios can be outperformed by more complicated multi-period expected utility maximizing strategies that are estimated using consensus analysts recommendations. This is not the case for \cite{he2013}, who find no significant portfolio returns from a \cite{black1992} strategy incorporating analyst recommendations for the Australian stock market.\\
%
\indent The evidence for return predictability from target prices is even more mixed: While \cite{brav2003} show that target prices are generally informative about future stock prices and that there are substantial short-term market reactions in the stock price to target changes, \cite{bonini2010} find little evidence for target price accuracy using different metrics measuring the prediction error between the target price and the current, twelve months forward and in-between spot prices. This is in line with \cite{bradshaw2012b} who find an average target price premium over the spot price of 15 percent and who report that only two thirds of the target prices are met by the stock price at some time during the forecast horizon in their sample. \cite{lin2016} instead consider changes in target prices, but also find no evidence that institutional trading activity following the direction of the target price revisions yields higher risk-adjusted out-of-sample returns. Only \cite{da2011} find that aggregating stocks across sectors according to their twelve month forward target price implied expected return, i.e. simple return between the current and the target price, yields significant risk-adjusted abnormal returns for different long-short portfolio.\\
%
\indent However, despite this discouraging evidence, none of the studies above operate in a Bayesian framework using predictive distributions and none consider a tilting framework in which the analyst's forecasts serve as a shrinkage target instead of just being another predictor variable.  

\subsection{Predictive regressions}
\cite{kandel1987} propose a vector autoregression formulation to jointly model the dynamics of asset returns and its predictor:
\begin{align}
	\label{eqn:ks1987}
	r_{t+1}&=a_r+b_rx_t+\varepsilon_{r,t+1},\\
	x_{t+1}&=a_x+b_xx_t+\varepsilon_{x,t+1},\label{eqn:ks1987_2}
\end{align}
where $x_t$ is the explanatory variable (for example the dividend yield) and $r_t$ is the asset return and $\varepsilon_t=(\varepsilon_{r,t},\varepsilon_{x,t})'\stackrel{iid}{\sim}\No{0,\Sigma}$. Here, equation (\ref{eqn:ks1987_2}) is needed to model the times-series dynamics of the predictor variable trough an autoregressive process. This model has been used by various authors to investigate the predictability of the equity premium and to build portfolios using the predictive return distribution.\\ 
%
%\indent \cite{campbell1988} uses it to show that the market equity premium can be predicted by the dividend yield as the predictor variable. Since then, the literature has provided evidence and counter-evidence for return predictability.\\
%
\indent \cite{kandel1996} estimate the model using Bayesian priors that center the mean of the slope coefficient in (\ref{eqn:ks1987}) to zero, implying no predictability and a weak form of market efficiency. Using a one-period model with power utility, they find that not accounting for the variability in the regressor can decrease the annual certainty equivalent up to 3\%.\\
%
\indent \cite{pastor2000a} extends the model in (\ref{eqn:ks1987}) - (\ref{eqn:ks1987_2}) to a multivariate regression for $N$ risky assets and apply a Normal inverse Wishart prior for the model mean and variance of the asset returns. They interpret the prior as a form to describe the disbelief of the investor in the given prediction/asset-pricing model. Importantly, they find that the optimal portfolio weights tend to be less extreme when using the joint predictive distribution of the asset returns by integrating out the estimated model parameters.% $\mu$ and $\Sigma$. %Similarly, \cite{pastor2000b} find that the difference in the portfolio selection based on various asset models vanishes when the degree of belief about their correctness decreased. %Eventually, \citet{wachter2009} model the investors disbelief in predictability through an informative prior on the  $R^2$ of the predictive regression.
\\
%
\indent While these studies are concerned with investments for one period, \cite{barberis2000} investigates the impact of changes in the prediction variables to the asset allocation in the long-term. As the variance of the cumulative log returns grows less than linear in the investment horizon, stocks appear less risky and so an investor should allocate more funds to stocks when the investment horizon increases. However such an over-allocating to stocks disappears when accounting for estimation uncertainty and possible structural breaks in the parameters of the prediction model \citep{pettenuzzo2011}.\\
%
\indent Stock returns may be predicted by company specific characteristics such as the dividend-price ratio or they may be macroeconomic indicators. Given the great number of potential predictor variables, shrinkage and model averaging methods are natural candidates to reduce model risk in predictive regressions. For example, \citet{avramov2002} considers 14 predictor variables and extend the predictive system in (\ref{eqn:ks1987}) - (\ref{eqn:ks1987_2}). The author averages over all $2^{14}=16384$ distinct model to predict the asset returns by weighting each model by its posterior weight. Similar examples for Bayesian model averaging (BMA) in the literature include \citet{cremers2002} and \citet{wachter2015}. Furthermore, \citet{pettenuzzo2016} propose a Bayesian density combination approach over 15 predictor variables that, instead of minimizing a statistical loss function, maximizes the certainty equivalent of a power utility investor and report robust out-of-sample predictability across various performance measures.\\
%
\indent Studies using time-varying coefficients in prediction models have also become numerous in recent years: While \cite{welch2008} find no evidence for significant in-sample and out-of-sample predictability of 14 variables on the equity premium using a rolling window estimation, \cite{dangl2012} find predictability in one-step-ahead forecasts of monthly S\&P 500 returns by using Bayesian predictive regressions with time-varying parameters and model averaging over a range of predictor variables. Using time-varying coefficients as well as stochastic volatility but only two possible predictor variables, \cite{johannes2014} document statistically and economically significant portfolio benefits for a dynamic rebalancing power utility investor using Bayesian predictive regressions. Very recently, \cite{feng2016} propose to perform a \textit{predictive} cross-validation approach to select the right amount of shrinkage that provides optimal posterior point estimates instead of marginalizing out the model parameters to obtain a full predictive return distribution. \\
%
\indent Very related to this paper is \cite{metaxoglou2016}, who are the first to apply entropic tilting to predict the US equity premium. For this, the authors rely on option data to form corresponding  moment conditions. In particular, the authors use the second moment of the risk-neutral density implied by option prices on the S\&P500 to tilt predictive distributions obtained from OLS model combinations with individual predictors, where the combination weights are chosen such that the model minimizes a discounted mean squared forecast error \citep{rapach2013}. For monthly returns and one-step ahead forecasts, they find that the tilted forecasts outperform an historical average based on various performance measures and for various sub-samples.
%
%\subsection{US equity premium predictability}
%\cite{ang2007} find significant in-sample predictability of the dividend yield, the short-term interest rate and the earnings yield on the equity premium (measured by S\&P500 excess returns) only for short horizons and only when excluding the 1990ies. While the predictive power of the dividend yield increases when including the short-term rate, which could be explained by an omitted variable bias, the predictive power of the earnings yield is very similar to the dividend yield, indicating that mostly the denominator (the price of the asset) has predictive power.\\
%\indent Examples that focus on varying coefficients in asset prediction models are among many other \cite{lettau2008}, \cite{welch2008}, \cite{dangl2012}, and \cite{johannes2014}.\\
%%
%\indent \cite{welch2008} find no evidence for significant in-sample and out-of-sample predictability of 14 variables on the equity premium measured by the excess return of the S\&P500 index. Besides considering individual predictive regressions for every predictor, they also consider a model including all variables and a model performing subset selection based on a cumulative predictive error criterion in every period. While the model including everything delivers significant in-sample predictability, it can not outperform the historical average out-of-sample. The model selection approach only delivers good performance until 1976, a couple of years after the oil price. The authors stress the point that excluding post-crisis periods (such as 1973-1975) deteriorates the performance of the many of the predictor variables against the historical average even further. The results hold for longer and shorter forecast horizons as well as for sub-sample periods.\\
%%
%\indent \cite{dangl2012} find predictability in a Bayesian regression setting with time-varying coefficients, but no stochastic volatility, and a range of predictor variables for a one-step-ahead forecast horizon. They apply a Bayesian model averaging approach that outperforms single regressor predictions. In connection with the business cycle, they show that out-of-sample predictability is stronger in recessions than in expansions. This can be used for market timing in portfolio allocation.\\
%%
%\indent Further in an expected utility framework with dynamic portfolio rebalancing, \cite{johannes2014} investigate the portfolio performance from predictive regressions over time. They authors use a prediction model that incorporates payout yield-based expected return predictors with time-varying coefficients and stochastic volatility. They estimate the model sequentially through partical filters to simulate a Bayesian learning problem for the investor about the parameters. They find statistically and economically significant out-of-sample portfolio benefits for a power utility investor who uses such models and accounts for parameter uncertainty by using predictive return distributions ($p(r_{t+h}|r_t)$) of return predictability when forming optimal portfolios.\\
%%
%\indent Using the VAR return prediction model of \cite{kandel1996} given in equation (\ref{eqn:ks1996}), \cite{feng2016} use the duality of regularization penalties (such as a lasso prior) on the mean and variances of asset returns and prior distributions to propose a \textit{predictive} cross-validation approach to select the right amount of shrinkage. While they are finding the optimal posterior point predictions, their approach does not marginalize out the parameters to obtain a full predictive return distribution.\\
%%
%\indent \cite{pettenuzzo2016} propose a Bayesian density combination approach that maximizes the certainty equivalent of a power utility investor who chooses between the market index (S\&P500 returns) and the risk free investment. They consider 15 predictor variables in individual models with constant parameters and with time-varying parameters and stochastic volatility. They find robust out-of-sample predictability across performance measures and also test predictability for Fama/French industry portfolios.\\
%%
%\indent \cite{metaxoglou2016} are the first to apply entropic tilting to predict the US equity premium. For this, the authors rely on option data to form corresponding  moment conditions. In particular, the authors use the second moment of the risk-neutral density implied by option prices on the S\&P500 to tilt predictive distributions obtained from OLS model combinations with individual predictors.  The combination weights are chosen such that the model minimizes a discounted mean squared forecast error (DMSFE) \citep{rapach2013}. For monthly returns and one-step ahead forecasts, they find that the tilted forecasts outperform an historical  based on various performance measures and for various sub-samples.
%
%\subsection{Cross-sectional stock return predictability}
%
%Examples are \cite{avramov2002} and \cite{rapach2015}. For a detailed review the reader is guided to \citet[][Section 4.2., page 374]{rapach2013}.

% to old to be true
%The model can be written in matrix form for $k$ explanatory variables using some zero restrictions:
%\begin{equation}
%\label{eqn:ks1987matrix}
%\begin{pmatrix} y_2'\\ \vdots \\ y_T' \end{pmatrix} = \begin{pmatrix} 1 & x_1'\\ 1 & \vdots \\ 1 & x_{T-1}' \end{pmatrix}\,B+\begin{pmatrix} \varepsilon_2'\\ \vdots \\ \varepsilon_T' \end{pmatrix}\,\Longleftrightarrow\,Y=XB+E,
%\end{equation}
%where $Y$ is a matrix of dimension $(T-1)\times(k+1)$, $X$ of $(T-1)\times(k+1)$, $B$ of $(k+1)\times(k+1)$, and $E$ of $(T-1)\times(k+1)$. For all $t$ it holds that $\varepsilon_t=(\varepsilon_{r,t},\varepsilon_{x,t})'\stackrel{iid}{\sim}\N{0,\Sigma}$.\\
%%
%\indent The model in (\ref{eqn:ks1987}) as been used by various authors to investigate portfolio problems: \cite{kandel1996} use a Bayesian version of it and investiate the predictive power and portfolio gains from using the dividend yield as the predictor variable. The argument is that, although the typical coefficient of determination for each equation in the system is lower than 10\%, the impact of changes in the regressor variable can have a sizable impact on the optimal asset allocation. They use a one-period model with power utility and an informative prior specification that centers the mean of the slope coefficient to zero, which implies a weak form of market efficiency. They find that not accounting for the variability in the regressor can decrease the annual certainty equivalent up to 3\%.\\
%
%proposed the model, later to be used by \cite{pastor2000a} and \cite{avramov2002} as well as many others for return predictions, for using the dividend yield to forecast asset returns. They propose a Bayesian vector autoregression (BVAR) model of order 1 for the asset returns and the dividend yield as follows:
%The second equation is an AR(1) process. As in the previous chapter, the model can be extended to a model with $k$ explanatory variables and some zero restrictions. In matrix form this is
%\begin{equation}
%\label{eqn:ks1996matrix}
%\begin{pmatrix} y_2'\\ \vdots \\ y_T' \end{pmatrix} = \begin{pmatrix} 1 & x_1'\\ 1 & \vdots \\ 1 & x_{T-1}' \end{pmatrix}\,B+\begin{pmatrix} \varepsilon_2'\\ \vdots \\ \varepsilon_T' \end{pmatrix}\,\Longleftrightarrow\,Y=XB+E,
%\end{equation}
%where $Y$ is a matrix of dimension $(T-1)\times(k+1)$, $X$ of $(T-1)\times(k+1)$, $B$ of $(k+1)\times(k+1)$, and $E$ of $(T-1)\times(k+1)$. For all $t$ it holds that $\varepsilon_t=(\varepsilon_{r,t},\varepsilon_{x,t})'\stackrel{i.i.d.}{\sim}\mathcal{N}(0,\Sigma)$. With this model, the authors investigate the impact of predictability on the optimal portfolio. The argument is that, although the typical coefficient of determination for each equation in the system is lower than 10\%, the impact of changes in the regressor variable can have a sizable impact on the optimal asset allocation. They use a one-period model with power utility and an informative prior specification that centers the mean of the slope coefficient to zero, which implies a weak form of market efficiency. They find that not accounting for the variability in the regressor can decrease the annual certainty equivalent up to 3\%.\\
%%
%\indent \cite{pastor2000a} investigates the effect of mixed beliefs of the investor about the true asset pricing model and about his own subjective return prior assumptions. With $N$ assets, the author specifies a multidimensional regression with the normal inverse Wishart prior for $\mu$ and $\Sigma$ as an extension to the model of \cite{kandel1996} which is only used for one risky asset. It can also be written in SURE form with a constant term in each equation. The variance of the prior distribution of this constant, which is centered around zero for all assets in order to reflect the correct specification of the pricing model, describes the investors' belief or disbelief in it. For the one period investment, in which the investor maximizes the Sharpe ratio, the author shows that the chosen optimal portfolio weights tend to be less extreme and less sensitive to estimation errors. In a similar fashion with a possible shrinkage interpretation, \cite{pastor2000b} compare the impact of believing in different asset pricing models and find that the difference in the portfolio selection based on various asset models vanishes when the degree of belief about the correctness and the predictive ability of one particular model decreases.\\
%%
%\indent While \cite{kandel1996} are concerned with investments for one period, \cite{barberis2000} investigates the impact of changes in the prediction variables in the long-term. Using the same model with the dividend yield as the predictor and the same prior specification, the author considers a buy-and-hold investor as well as an investor who rebalances dynamically in each period. The author rewrites the system in (\ref{eqn:ks1996}) to 
%\begin{equation}
%z_t=(r_t,x_t)'=\alpha+B_0z_{t-1}+\varepsilon_t,
%\end{equation}
%where $\alpha=(\alpha_r,\alpha_x)'$ and $B_0=\begin{pmatrix} 0 & \beta_r\\0 & \beta_x\end{pmatrix}$ and it follows that
%\vspace{-0.3cm}
%\begin{align}
%	\Rightarrow z_{T+1}&=\alpha+B_0z_T+\varepsilon_{T+1}\nonumber\\
%	z_{T+2}&=\alpha+B_0\mu+B-0^2z_T+B_0\varepsilon_{T+1}\varepsilon_{T+2}\nonumber\\
%	\vdots\nonumber\\
%	\vspace{-0.7cm}
%	z_{T+h}&=\alpha\sum_{i=0}^{h-1}B0^i+B_0^hz_T+\sum_{i=0}^{h-1}B_0^i\varepsilon_{T+h-i}.
%\end{align}
%Then  using the results of \citet[Appendix A]{avramov2002}, his argument for continuously compounded returns is that the predictive distribution of $Z_{T+h}=z_{T+1}+\ldots,z_{T+h}$ conditional on $\mathcal{Y}_T$ and the set of parameters of the Bayesian VAR model $\Theta$ is given by 
%\begin{align}
%	p(Z_{T+h}|\mathcal{Y}_T,\Theta)&\sim\mathcal{N}({\mu}_{T+h},{\Sigma}_{T+h})\\
%	\text{with}\quad{\mu}_{T+h}&=\mu\sum_{i=0}^{h-1}(h-i)B_0^i+z_T\sum_{i=1}^hB_0^i\\
%	\text{and}\quad{\Sigma}_{T+h}&=\Sigma+(I+B_0)\Sigma(I+B_0)+\cdots\nonumber\\
%	&+(I+B_0+B_0^2+\cdots+B_0^{h-1})\Sigma(I+B_0+B_0^2+\cdots+B_0^{h-1}).
%%\end{align} The author shows that an investor allocates more funds to stocks when the investment horizon increases. This holds for an investor who maximizes his expected utility in a Bayesian sense, but also when the investor uses plug-in estimates. The idea here is that the variance of cumulative log returns grows less than linear in the investment horizon and thus stocks appear less risky. When accounting for estimation risk, the investor is not only uncertain about the future mean return, but also doubts the predictive power of the dividend yield and that the variance of the long-term investment increases. It follows that the investor might over-allocate funds to stocks relative to the risk taken when not accounting for estimation risk. It changes when we allow for rebalancing. Then, the investor will allocate less to stocks when the investment horizon increases. The author explains this by an increasing hedging demand of the investor.\\
%%%
%\indent Model uncertainty of the predictive regression adds another layer risk for the expected utility maximizing investor. The choice of predictor variables can range from company-specific to economy-wide indicators and allows for endless combinations of different variables. A way to reduce model uncertainty is to use a model averaging approach as in \citet{avramov2002}. The author extends the Bayesian vector autoregressive framework with 14 different predictor variables in each equation which yields $2^{14}=16384$ model choices for the two equations in (\ref{eqn:ks1996}). The predictive return distribution can be calculated by averaging over the predictive return distributions of all models multiplied by their corresponding posterior model probabilities as in section \ref{sec:bma}. Here, the use of MCMC methods is of utmost importance because the marginal predictive distributions of each model are usually analytically intractable and the number of possible models is very large. \citet{avramov2002} rewrites the multivariate model as seemingly unrelated regressions with normal-inverse Wishart priors within each return model. In addition, each model has the same prior probability. The author then shows that their BMA approach outperforms all models chosen by the highest odd ratios and that the impact of model risk can be at least as high as the impact of estimation risk.\\
%
%Studies of asset allocation under return predictability (e.g., Barberis
%(2000), Campbell and Viceira (2001), Campbell et al. (2003)
%and Kandel and Stambaugh (1996)) have mostly used vector autoregressions
%(VARs) to capture the relation between asset returns
%and predictor variables. We follow this literature and focus on a
%simple model with a single risky asset and a single predictor variable.
%This gives rise to a bivariate model relating returns (or excess
%returns) on the risky asset to a predictor variable, xt . Empirically,
%the coefficients on the lagged returns are usually found to be small,
%so we follow common practice and restrict them to be zero. The resulting
%model takes the form
%%%zt = B′ ˜xt−1 + ut , (1)
%%where zt = (rt , xt )′ , ˜xt−1 = (1, xt−1)′ , rt is the stock return at time
%%t in excess of a short risk-free rate, while xt is the predictor variable
%%and ut ∼ IIDN(0,Σ), whereΣ = E[utu′
%%t ] is the covariance matrix.
%%We refer to μr and μx as the intercepts in the equation for the
%%return and predictor variable, respectively, while βr and βx are the
%%coefficients on the predictor variable in the two equations:
%%rt = μr + βr xt−1 + urt
%%xt = μx + βxxt−1 + uxt .
%
%\indent Model uncertainty of the predictive regression adds another layer risk for the expected utility maximizing investor. The choice of predictor variables can range from company-specific to economy-wide indicators and allows for endless combinations of different variables. A way to reduce model uncertainty is to use a model averaging approach as in \citet{avramov2002}. The author extends the Bayesian vector autoregressive framework with 14 different predictor variables in each equation which yields $2^{14}=16384$ model choices for the two equations in (\ref{eqn:ks1996}). The predictive return distribution can be calculated by averaging over the predictive return distributions of all models multiplied by their corresponding posterior model probabilities as in section \ref{sec:bma}. Here, the use of MCMC methods is of utmost importance because the marginal predictive distributions of each model are usually analytically intractable and the number of possible models is very large. \citet{avramov2002} rewrites the multivariate model as seemingly unrelated regressions with normal-inverse Wishart priors within each return model. In addition, each model has the same prior probability. The author then shows that their BMA approach outperforms all models chosen by the highest odd ratios and that the impact of model risk can be at least as high as the impact of estimation risk.\\
%%
%\indent \cite{ravazzolo2008} propose a Bayesian model averaging approach to conquer estimation, model and parameter risk. They propose the following return prediction model:
%\begin{align}
%	r_t&=\beta_{0,t}+\sum_{j=1}^ks_j\beta_{j,t}x_{j,t}+\varepsilon_t\\
%	\beta_{j,t}&=\beta_{j,t-1}+\kappa_{j,t}\eta_{j,t},\quad j=1,\ldots,k,
%\end{align}
%where $\varepsilon_t\sim\mathcal{N}(0,\sigma^2)$, $s_j$ is an indicator variable about including predictor variable $j$ or not and $\beta_t=(\beta_{0,t},\ldots,\beta_{k,t})$ are time-variable coefficients. The second equation is a random walk with $\eta_{j,t}\sim\mathcal{N}(0,\sigma_{j,t}^2)$ and $\kappa_{j,t}$ is a variable  indicating whether a structural break has occurred for parameter $j$ at time $t$ or not. They use a Beta distribution for the prior probability of a structural break and set up a hierarchical Gibbs sampler. They show for monthly S\&P500 returns that an active investment strategy with dynamic rebalancing and a forecasting model accounting for model and parameter uncertainty is worth several hundred basis points annually in comparison to a passive buy-and-hold strategy.\\
%\indent Many economic relations change over time, also the relationship between predictor variables and the stock returns. Henceforth, structural breaks can also have a major impact on the predictability of the return process and on the portfolio performance. These relations may shift smoothly and constantly over time, or the model may exhibit large structural breaks that seldom appear. Exemplary, \citet{pettenuzzo2011} adopt the multiple change point model proposed by \citet{chib1998} and \citet{pastor2001} and focus on the Bayesian first-order vector autoregressive (VAR) formulation in (\ref{eqn:ks1996}) for the return and predictor variable process. In order to consider structural breaks, they use an hidden Markov model (HMM) in which predictive return distribution is conditional on a integer-state variable $S_t$ which tracks individual regimes of model parameter shifts. They model the state variable by a hidden Markov chain. They find that a risk-averse buy-and-hold investor who accounts for parameter instability decreases his proportion of wealth invested in stocks when the investment horizon increases.
%5While Welch and Goyal (2008) comprehensively examine fourteen predictor variables
%and find little forecasting power in univariate forecasting regressions, Cochrane
%(2008) uses a VAR system of returns and dividend growth to explore their joint stochastic
%dynamics and defend the return predictability. Furthermore, Rapach et al. (2010), Li
%and Tsiakas (2015) and Hull and Qiao (2015) all show evidence of using a combination
%of multiple predictors outperforming univariate forecasting regressions. 
%Stambaugh (1999) provides
%a detailed analysis of Bayesian predictive regressions and illustrates the drawback
%of OLS when using lagged stochastic regressors.
%
%\indent \citet{wachter2009} consider a mean-variance investor with quadratic utility who can invest into a risk-free asset, a long-term bond and a risky stock index. As predictor variables, the authors consider the dividend yield and the corporate yield spread. They use a normal prior for the predictor variable coefficients and a diffuse prior for all other parameters. In addition, they model the investors skepticism about the degree return predictability by an informative prior over the regressions: the $R^2$. For monthly returns with quarterly rebalancing for the time period from 1972 to 2005, they find evidence that any investor should rebalance his portfolio on the predictor variable basis and that the resulting asset weights are less volatile and deliver a superior out-of-sample performance as compared to the weights implied by an entirely model-based or data-based view.\\
%
%
