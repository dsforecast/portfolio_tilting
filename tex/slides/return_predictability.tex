\documentclass[9pt,xcolor=x11names,compress]{beamer}

%%% Author and Title Information %%%
\newcommand{\titleinfo}{Sequential Stock Return Prediction Through Copulas}
\title{\titleinfo}
\def\authorA{Christoph Frey}
\def\affiliationA{Erasmus University Rotterdam} 
\def\emailA{\href{mailto:frey@ese.eur.nl}{frey@ese.eur.nl}}
\def\authorB{Audra Virbickait\.{e}}
\def\affiliationB{University of Konstanz} 
\def\emailB{\href{mailto:virbickaite.audrone@uni.kn}{virbickaite.audrone@uni.kn}}
\newcommand{\authorinfo}{\authorA, \authorB}

%%% General Packages
\usepackage[english]{babel} 
\usepackage[OT1]{fontenc}
\usepackage[utf8]{inputenc}
\usepackage{color}
\usepackage{graphicx}
\usepackage{caption}

%%%%%%%%%%%%%%
%%% Beamer Layout %%%
%%%%%%%%%%%%%%

%%% Default Theme
\usetheme{default}

%%% Font Family
\usepackage{cmbright}

%%% No Navigation Bar
\setbeamertemplate{navigation symbols}{}

%%% Colors
\definecolor{dblue}{RGB}{40,40,110} %blue
\definecolor{dgreen}{RGB}{10,140,70} %green
\definecolor{dred}{RGB}{180,30,30} %red

\setbeamercolor{frametitle}{fg=dblue}
\setbeamercolor{button}{bg=dblue,fg=white}
\setbeamertemplate{section in head/foot}{\textcolor{dblue}{\hfill\insertsectionhead}}
\setbeamertemplate{section in head/foot shaded}{\textcolor{black!15}{\hfill\insertsectionhead}} % faded section titles for inactive sections

% Colored Links
\usepackage{hyperref}
\hypersetup{colorlinks=true,citecolor=dblue,pdfpagelabels=TRUE,pdfstartview={FitH},linkcolor=dblue,urlcolor=dblue}

%%% Headline
\setbeamerfont{headline}{size=\fontsize{6.5}{1},shape=\scshape}
\setbeamertemplate{headline}{
	\vspace{0.015\paperheight}
	{\leavevmode
		\begin{beamercolorbox}{section in head/foot} 
			\insertsectionnavigationhorizontal{\paperwidth}{\hspace*{0.1ex}}{\hspace*{0.1ex}}
		\end{beamercolorbox}%
		\vspace{0.005\paperheight}}
	\begin{beamercolorbox}[center]{}
		\hspace{0.025\paperwidth}\rule{0.95\paperwidth}{0.5pt}\hspace{0.025\paperwidth}
	\end{beamercolorbox}%
}

%%% Footline
\setbeamerfont{footline}{size=\fontsize{6.5}{1},shape=\scshape}
\setbeamertemplate{footline}{
	\begin{beamercolorbox}[center]{}
		\hspace{0.025\paperwidth}\rule{0.95\paperwidth}{0.5pt}\hspace{0.025\paperwidth}
	\end{beamercolorbox}%
	\leavevmode
	\begin{beamercolorbox}[wd=.5\paperwidth,ht=2.8ex,left]{}
		\hspace*{0.025\paperwidth}\textcolor{dblue}{\insertsubsectionhead}
	\end{beamercolorbox}%
	\begin{beamercolorbox}[wd=.5\paperwidth,ht=2.8ex,right]{}
		\insertframenumber{} $|$ \inserttotalframenumber \hspace{0.025\paperwidth} 
	\end{beamercolorbox}%
	\vspace{0.015\paperheight}
}

%%% Itemize / Enumerate Symbols
\setbeamertemplate{itemize items}{\textcolor{dblue}{\raisebox{0.6pt}{\footnotesize$\blacktriangleright$}}}
\setbeamertemplate{itemize subitem}{\textcolor{dblue}{\raisebox{0.6pt}{\footnotesize$\triangleright$}}}
\setbeamertemplate{itemize subsubitem}{\textcolor{dblue}{\raisebox{0.6pt}{\footnotesize$\bullet$}}}
\setbeamertemplate{enumerate items}{\textcolor{dblue}{\raisebox{0.6pt}{\footnotesize\insertenumlabel.}}}

%%% Title Page
\setbeamertemplate{title page}{%
	\vspace*{0.1\paperheight}
	{\LARGE \color{dblue}{\inserttitle}}
	\vskip1em\par
	\normalsize\authorA
	\vskip0.5em\par
	\hspace{0.3cm} \footnotesize\affiliationA\vskip1mm\par
	\hspace{0.3cm} \footnotesize\emailA\vskip1mm\par
	\vskip1em\par
	\normalsize\authorB
	\vskip0.5em\par
	\hspace{0.3cm} \footnotesize\affiliationB\vskip1mm\par
	\hspace{0.3cm} \footnotesize\emailB
	\vspace{1.8cm}
	\begin{figure}
		\today\hfill\includegraphics[height=0.125\linewidth]{signature.pdf}\hspace{.2cm}\includegraphics[height=0.125\linewidth]{unikn.pdf}
	\end{figure}
}

%%% New Commands

% Text Color Commands
\newcommand{\btext}{\textcolor{dblue}}
\newcommand{\bbtext}[1]{ \textcolor{dblue}{\textbf{#1}}}
\newcommand{\rtext}{\textcolor{dred}}
\newcommand{\rbtext}[1]{ \textcolor{dred}{\textbf{#1}}}
\newcommand{\gtext}{\textcolor{dgreen}}
\newcommand{\gbtext}[1]{\textcolor{dgreen}{\textbf{#1}}}

% Enumerate Counter
\newcounter{saveenumi}
\newcommand{\seti}{\setcounter{saveenumi}{\value{enumi}}} % save current counter
\newcommand{\conti}{\setcounter{enumi}{\value{saveenumi}}} % continue current counter

% Set Introduction = section 0
\setcounter{section}{-1}

% Bibliography
\usepackage{hyperref}
\usepackage{nameref}
\usepackage[labelformat=empty]{caption}
\usepackage[sectionbib]{natbib}
\let\natbibcitet\citet
\renewcommand\citet{\bibpunct{(}{)}{,}{a}{,}{,}\natbibcitet}
\let\natbibcitep\citep
\renewcommand\citep{\bibpunct{(}{)}{;}{a}{,}{;}\natbibcitep}

\let\oldcite=\cite                                                              
\renewcommand{\cite}[1]{\textcolor{dblue}{\oldcite{#1}}}

%\let\oldcitet=\citet                                                            
%\renewcommand{\citet}[1]{\textcolor{dblue}{\oldcite{#1}}}


\begin{document}
\section{Introduction}
\begin{frame}[plain]
\titlepage
\end{frame}
%%%%%%%%%%%%%%%%%%%%%%%%%%%
%%%%%%%%%%%%%%%%%%%%%%%%%%%

%\begin{frame}{Content}
%\tableofcontents
%\end{frame}
%%%%%%%%%%%%%%%%%%%%%%%%%%%
%%%%%%%%%%%%%%%%%%%%%%%%%%%

\begin{frame}{Motivation}
	\begin{itemize}
		\item Predicting asset returns is a popular exercise in finance
		\pause
		\item Evidence for predictability when accounting for model and estimation uncertainty: model averaging \cite{pettenuzzo2016}, time-varying-parameters \cite{dangl2012}, stochastic volatility and predictive distributions \cite{Johannes2014}
		\pause
		\item But: Linear models, limited set of lagged predictor variables
		\pause
		\item Evidence for (weak) rolling window serial correlation in return series
	\end{itemize}
\vspace{-0.4cm}
\begin{figure}
\centering\label{fig:rolcol}
\includegraphics[width=0.7\linewidth]{rolcol}
\end{figure}
%		
%		
%		using dynamic model  
%		
%		\item In this work we build on models described in Chen and Fan (2006) and Bouye et al.
%		(2002), however, instead of time-invariant copula we consider dynamic dependence structure.
%		We argue, that even though nancial returns exhibit no auto-regressive behavior (in a
%		classical sense), there is considerable amount of correlation (linear and rank) if we consider
%		time-varying dependence. In other words, there is correlation present between log-returns
%		and lagged log-returns and it is not constant.
%		\item equity premium returns are predictable when using Evidence for return
%		\item Return predictability is a stylized fact 
%		\begin{itemize}
%			\item \btext{\citet[][page 842]{Lettau2001}}: \textit{``It is now widely accepted that excess returns are predictable by variables such as dividend price ratios, earnings-price ratios, dividend-earnings ratios, and an assortment of other financial indicators.''}
%			\item \cite{Johannes2014} report portfolio benefits only from using a model with time-varying coefficients, stochastic volatility model and that accounts for parameter uncertainty:
%			\begin{align*}
%			r_{t+1}&=\alpha+\beta_{t+1}x_t+\sqrt{V_{t+1}^r}\varepsilon_{t+1}^r\\
%			\beta_{t+1}&=\alpha_{\beta}+\beta_{\beta}\beta_t+\sigma_{\beta}\varepsilon_{t+1}^{\beta},\quad\varepsilon_{t+1}^{\beta}\stackrel{iid}{\sim}\text{N}(0,1)\\
%			x_{t+1}&=\alpha_x+\beta_xx_t+\sqrt{V_{t+1}^x}\varepsilon_{t+1}^x\\
%			\log{V_{t+1}}&=\alpha_v+\beta_v\log{V_{t}}+\sigma_v\eta_{t+1}^v,\quad\eta_{t+1}^v\stackrel{iid}{\sim}\text{N}(0,1)
%			\end{align*} 
%			\item Volatility timing effect?
%		\end{itemize}
%	\end{itemize}
\end{frame}
%%%%%%%%%%%%%%%%%%%%%%%%%%%
%%%%%%%%%%%%%%%%%%%%%%%%%%%

%\begin{frame}{Rolling window serial correlation in return series}
%			\begin{itemize}
%				\item (Weak) rolling window serial correlation in return series
%			\begin{itemize}
%				\item Rolling window has to be \textit{sufficiently} small
%				\item Relation to momentum literature in finance
%			\end{itemize}
%		\end{itemize}
%		\vspace{-0.4cm}
%	\begin{figure}
%		\centering\label{fig:rolcol}
%		\includegraphics[width=0.9\linewidth]{rolcol}
%	\end{figure}
%\end{frame}
%%%%%%%%%%%%%%%%%%%%%%%%%%%%
%%%%%%%%%%%%%%%%%%%%%%%%%%%%

\begin{frame}{}
\begin{itemize}
	\item Idea: Model the temporal dependence between a univariate financial asset and its past by specifying 
	\begin{enumerate}
		\item[(i)] an invariant distribution $F$ and
		\item[(ii)] a bivariate copula $C$ that characterizes the dependence between consecutive realizations and
		\item[(iii)] that is governed by a time-varying copula parameter that can describe the dynamic auto-correlation in the asset returns.
	\end{enumerate}
	\item \cite{Chen2006}: In other words, model a univariate stationary discrete-time Markov chain for the return series
\end{itemize}
\begin{itemize}
	\item[$\to$] Avoid additional variables
	\item[$\to$] Model nonlinear dependence: A copula that exhibits tail dependence may generate a Markov chain which appears to become substantially more serially dependent as it draws towards the extremes of the state space.
\end{itemize}
\end{frame}
%%%%%%%%%%%%%%%%%%%%%%%%%%%
%%%%%%%%%%%%%%%%%%%%%%%%%%%

%\begin{frame}{Conditional Means and conditional standard deviations}
%	\begin{figure}
%		\centering\label{fig:patton}
%		\includegraphics[width=1\linewidth]{patton}
%	\end{figure}
%	\vspace{-0.5cm}
%	\begin{itemize}
%		\item Conditional mean $\pm$ conditional standard deviation for joint distribution with N(0,1) marginals and correlation coefficient 0.5
%	\end{itemize}
%\end{frame}
%%%%%%%%%%%%%%%%%%%%%%%%%%%%
%%%%%%%%%%%%%%%%%%%%%%%%%%%%


\begin{frame}{Literature Review - Copulas for univariate time series}
	\begin{itemize}
		\item \cite{darsow1992}, \cite{ibragimov2008}, \cite{ibragimov2009}, \cite{Beare20010} explore the relationship between Markov processes and a copula function for univariate time series
		\item \btext{\citet[][Chapter 8]{Joe1997}} studies Markov processes using parametric copulas with constant copula parameter and parametric marginals
		\item \cite{Chen2006} and \cite{Chen2009} study univariate copula-based semi-parametric stationary Markov models, in which they use parameterized copulas empirical marginal distributions
		\item \cite{Abegaz2008} propose a dynamic copula model for a Markov chain with a non-random ARMA-like specification for the evaluation of the Copula parameter
		\item \cite{sokolinskiy2011} forecast daily volatility through a semi-parametric copula realized volatility model, but disregard forecasting power of a deterministic time-varying copula parameter	
	\end{itemize}
\end{frame}
%%%%%%%%%%%%%%%%%%%%%%%%%%%
%%%%%%%%%%%%%%%%%%%%%%%%%%%

\begin{frame}{Contribution}
	\begin{itemize}
		\item We consider a\bbtext{time-varying stochastic copula} to model time-varying dependence between a single financial asset and its past. 
		\item We show that \rbtext{daily returns} exhibit time-varying autoregressive behavior which is captured through the copula. 
		\item The daily returns are standardized by\bbtext{realized volatility}.
		\item The use of \rbtext{asymmetric copulas} enables us to capture asymmetric densities and non-linear dependencies within the return series. 
		\item The copula parameter here follows a first order autoregressive process and is estimated using \bbtext{sequential Monte Carlo} techniques. 
		\item We find that dynamic copulas produces \rbtext{better out-of-sample density forecasts} in terms of Bayes factors and log-predictive tail scores than benchmark models from the literature. 
	\end{itemize}
\end{frame}
%%%%%%%%%%%%%%%%%%%%%%%%%%%
%%%%%%%%%%%%%%%%%%%%%%%%%%%



\begin{frame}{Copula - General concept}
	\begin{itemize}
		\item Let $X$ and $Y$ be two random variables with distribution functions $F(x)=P[X\leq x]$ and $G(y)=P[Y\leq y]$ and a joint distribution function $H(x,y)=P[X\leq x, Y\leq y]$. Then, according to a theorem by \cite{Sklar1959}, there exists a copula $C$ such that 
		\begin{equation}
		H(x,y) = C(F(x),G(y))
		\end{equation}   
		\item The copulas is defined on the unit hypercube $[0,1]^d$ where all univariate marginals are distributed as $\mathcal{U}(0,1)$.
		\item Let $u=F(x)$ and $v=G(y)$ and $\theta$ is the copula parameter. The density of a bivariate copula is $c(u,v;\theta)=\partial^2 C(u,v;\theta)/\partial u \partial v$, whereas the density of the joint distribution is 
		\begin{equation}
		h(x,y)=c(F(x),G(y))\cdot f(x)\cdot g(y)
		\end{equation}
	\end{itemize}
\end{frame}
%%%%%%%%%%%%%%%%%%%%%%%%%%%
%%%%%%%%%%%%%%%%%%%%%%%%%%%

%\begin{frame}{Copula - General concept}
%	Copulas are very flexible in the sense that 
%		\begin{itemize}
%			\item marginal distributions can be modeled independently from the dependence structure
%			\item they are able to capture asymmetric non-linear dependencies, as opposed to the standard multivariate distributions, such as Gaussian
%		\end{itemize} 
%\end{frame}
%%%%%%%%%%%%%%%%%%%%%%%%%%%
%%%%%%%%%%%%%%%%%%%%%%%%%%%

\section{Methodology}

\begin{frame}{Model and Estimation}
	\begin{itemize}
		\item Denote the asset return at time $t$ by $r_t$.
		\item The joint density:
		\begin{equation*}
		h(r_t,r_{t-1}) = c(F(r_t), F(r_{t-1});\theta)f(r_t)f(r_{t-1}),
		\end{equation*}
		then, the conditional distribution of $r_t$ given $r_{t-1}$ (or $u_{t-1}$) is 
		\begin{equation*}
		f(r_t|r_{t-1}) =\frac{h(r_t,r_{t-1}) }{f(r_{t-1})} = c(F(r_t), F(r_{t-1});\theta)f(r_t) = c(u_t, u_{t-1};\theta)f(r_t)
		\end{equation*}
		such that $F(\cdot) = \Phi(\cdot)$ and $f(\cdot) = \phi(\cdot)$, cdf and pdf of standard Normal respectively.
	\end{itemize}
\end{frame}
%%%%%%%%%%%%%%%%%%%%%%%%%%%
%%%%%%%%%%%%%%%%%%%%%%%%%%%

\begin{frame}{Parametric marginal distributions}\label{rv}
	\begin{itemize}
		\item Consider demeaned  log-returns (in \%) of some financial asset:
		\begin{align*}
		r_t = 100\times\left(\log \frac{P_t}{P_{t-1}}-\text{E}\left[\log \frac{P_t}{P_{t-1}}\right]\right),
		\end{align*}
		\pause
		\item Let $RV_t$ as a realized \textit{ex post} volatility measure, obtained from 10-minute returns $RV_t = \sum_{i=1}^N \tilde{r}_{i,t}^2$, where $\tilde{r}_{i,t}$ is a 10-minute log-return %for day $t$ and $N$ is the number of 10-minute intervals in a trading day
		\pause
		\item Then, standardize demeaned returns via realized volatility measure: 
		\begin{align*}
		\hat{z}_t = r_t/\sqrt{RV_t},\:\:\:\text{such that}\:\:\:\hat{z}_t\stackrel{approx}{\sim}\mathcal{N}(0,1)\quad \hyperlink{sampling}{\beamergotobutton{QQ plot}}
		\end{align*}
		
		\pause
		\item Forecast $\log RV_t$ via AR(1) model
		\pause
		\item Complete model for the marginals:
		\begin{align}
		r_t &= \epsilon^{r}_t\sqrt{RV_t},\\
		\log(RV_t) &= \mu^{RV}+\phi^{RV}\log(RV_{t-1})+\tau^{RV}\epsilon^{RV}_t,
		\end{align}
		such that $\epsilon^r_t$ and $\epsilon^{RV}_t$ are uncorrelated $\stackrel{iid}{\sim}\mathcal{N}(0,1)$
\end{itemize}
%	\vspace{-0.2cm}
%	\begin{figure}
%		\centering
%		\includegraphics[width=0.8\linewidth]{rv}
%	\end{figure}
\end{frame}
%%%%%%%%%%%%%%%%%%%%%%%%%%%
%%%%%%%%%%%%%%%%%%%%%%%%%%%

\begin{frame}{Copulas}
	\begin{itemize}
		\item We consider only one-parameter copulas, such as Gaussian, Gumbel (upper tail dependence) and Clayton (lower tail dependence)
		\item \cite{Almeida2012}: Use Kendall's $\tau_{\kappa}=4\int\int_{I^2}C(u,v)d C(u,v)-1$ \citep[see][]{Nelsen2006} instead of $\theta$ in order to compare the dependence across different copulas 
		\item In order to be able to compare the dependence across different copulas using Kendall's $\tau_{\kappa}$, we need to make sure it lies in the same domain for all copulas of interest
		\item Therefore, consider double Gumbel ($c_{DG}$) and Clayton ($c_{DC}$) copulas, that are defined as follows:
		%Note, that for Gaussian copula $\tau_{\kappa} \in [-1,1]$, however, for standard Gumbel and Clayton copulas $\tau_{\kappa} >0$. Therefore, instead of considering standard Gumbel ($c_G$) and Clayton ($c_C$), we  use 
		\begin{align*}
		c_{DC}(u,v;\theta)&=\begin{cases}
		c_C(u,v;\theta) &\text{if }\theta\geq 0\\
		c_C(1-u,v;-\theta) &\text{if }\theta<0\end{cases}\\
		c_{DG}(u,v;\theta)&=\begin{cases}c_G(u,v;\theta+1) &\text{if } \theta\geq 0\\
		c_G(1-u,v;-\theta+1) &\text{if }\theta<0\end{cases}
		\end{align*}
	\end{itemize}
\end{frame}
%%%%%%%%%%%%%%%%%%%%%%%%%%%
%%%%%%%%%%%%%%%%%%%%%%%%%%%

\begin{frame}{Time-varying Stochastic Copula}
	\begin{itemize}
		\item Following a two-step estimation approach, we first estimate the marginals and using probability integral transform obtain $u_t = \Phi(\hat{z}_t)$ and  $u_{t-1} =\Phi(\hat{z}_{t-1})$. Then, if the densities are specified correctly $u_t, u_{t-1}\stackrel{iid}{\sim}\mathcal{U}(0,1)$. 
		\item Define the time-varying stochastic copula autoregressive model (SCAR) as follows \citep{Hafner2012}:
		\begin{align} \notag
		(u_t,u_{t-1})&\sim C\left((u_t,u_{t-1});\theta_t\right),\nonumber\\  \label{f:theta_ktau}
		\theta_t&=f_{\tau}(x_t),\\ 
		x_t&=\alpha+\phi x_{t-1}+\delta\eta_t, \:\: \eta_t\stackrel{iid}{\sim} \text{N}(0,1).\nonumber
		\end{align}
	\item $(u_t,u_{t-1})$ is a bivariate time series process that has a distribution function, defined here by a certain one-parameter copula. 
	\item The stochastic process $x_t$, that is unobserved and follows Gaussian autogressive process of order one with parameters $\Theta = (\alpha,\phi,\delta)$. 
	\end{itemize}
\end{frame}
%%%%%%%%%%%%%%%%%%%%%%%%%%%
%%%%%%%%%%%%%%%%%%%%%%%%%%%

\begin{frame}{Time-varying Stochastic Copula (cont.)}
	\begin{itemize}
		\item We model $x_t$ unconstrained and use $f_{\tau}$ to obtain $\theta_t$
		\item As $f_{\tau}(x_t)$ in \eqref{f:theta_ktau} is a copula-specific function, using the inverse Fisher z-transformation on the Normally distributed unobserved state variable $x_t$ yields Kendall's $\tau_{\kappa} \in (-1,1)$:
		\begin{equation*}
		\tau_{\kappa t} = \frac{\exp\{2x_t-1\}}{\exp\{2x_t+1\}}.
		\end{equation*}
		\item Then transform the obtained $\tau_{\kappa}$ to a copula parameter and obtain a state-space representation of the SCAR model:
%		\begin{align}\label{f:scar1}
%		(u_t,u_{t-1})|x_t &\sim C\left((u_t,u_{t-1});\theta_t\right),\:\:\text{where}\:\:\theta_t = f_{\tau}(x_t),\\\label{f:scar2}
%		x_t|\Theta, x_{t-1} &\sim \text{N}(x_t;\alpha+\phi x_{t-1},\delta^2).
%		\end{align}
		\begin{align}\label{f:scar1}
		\notag
		u_t& = \Phi(r_t/\sigma_t),\\
		(u_t,u_{t-1})|x_t &\sim C\left((u_t,u_{t-1});\theta_t\right),\:\:\text{where}\:\:\theta_t = f_{\tau}(x_t),\\
		x_t|\Theta, x_{t-1} &\sim \mathcal{N}(x_t;\alpha+\phi x_{t-1},\tau^2),\notag
		\end{align}
		\pause
		\item The predictive conditional density for one-step-ahead returns, given the past transformed standardized returns $u_t = \Phi(r_t/\sqrt{\sigma^2_t})$, is:
		{\footnotesize\begin{align*}
			f(r_{t+1}|u_{t}) = \int\int c\left(\Phi\left(\frac{r_{t+1}}{\sqrt{\sigma^2_{t+1}}}\right),u_t|\theta_{t+1}(x_{t+1}), \sigma^2_{t+1}\right)f(r_{t+1}|\sigma^2_{t+1}) f(x_{t+1}) f(\sigma^2_{t+1}) dx_{t+1} d\sigma^2_{t+1}
			\end{align*}}
	\end{itemize}
\end{frame}
%%%%%%%%%%%%%%%%%%%%%%%%%%%
%%%%%%%%%%%%%%%%%%%%%%%%%%%


\begin{frame}{Copula Estimation - Algorithm and priors}
	\begin{itemize}
		\item Perform sequential estimation for the SCAR model
		\item Use a modified version of Particle Learning of \cite{Carvalho2010}
		\item Use the sufficient statistics of \cite{Storvik2002} to allow for parameter learning
		\item Priors for copula parameters are chosen to be conditionally conjugate: $x_0\sim \mathcal{N}(c_0,C_0)$, 
		$\delta^2 \sim \mathcal{IG} (b_0/{2},b_0\delta_0^2/{2})$, 
		$\phi|\delta^2 \sim \mathcal{TN}(m_{\phi},V_{\phi}\delta^2)$ and  
		$\alpha\sim \mathcal{N}(m_{\alpha},V_{\alpha})$. 
		Here $\mathcal{TN}_{(a,b)}$
		represents Normal distribution, truncated at $a$ and $b$, while $c_0$, $C_0$,  $b_0$,
		$b_0\delta_0^2$, $m_{\phi}$, $V_{\phi}$, $m_{\alpha}$ and $V_{\alpha}$ are the hyper-parameters. 
		\item Priors for RV parameters are chosen to be conditionally conjugate: $\tau^{2(RV)}\sim \mathcal{IG} (b^{(RV)}_0/{2},b_0\tau_0^{2(RV)}/{2})$, $\phi^{(RV)}|\tau^{2(RV)} \sim \mathcal{TN}_{(-1,1)}(m^{(RV)}_{\phi},V^{(RV)}_{\phi}\tau^{2(RV)})$, and $
		\mu^{(RV)} \sim \mathcal{N}(m^{(RV)}_{\mu},V^{(RV)}_{\mu})$.
	\end{itemize}
\end{frame}
%%%%%%%%%%%%%%%%%%%%%%%%%%%
%%%%%%%%%%%%%%%%%%%%%%%%%%%

\begin{frame}{Sequential estimation for the SCAR model I}
	\begin{itemize}
		\item Estimate state-space model via particle filter for parameters in $\Theta$
		\item Set of sufficient statistics $S_t$ contains all updated hyper-parameters as well as filtered state variables $x_t$
		\item For $t=1\ldots,T$ and for each particle $(i)$  iterate through the following steps:
	\end{itemize} 
\begin{itemize}
	\item[1.]\textbf{(Blind) Propagating:}\\ Sample new  hidden states $x_{t+1}$ from $x_{t+1}\sim p(x_{t+1}|x_t,\Theta)
	$ and obtain $\theta_{t+1}$ deterministically.
	\item[2.]\textbf{Resampling:}\\ Resample old particles (parameters, sufficient statistics, states) with weights $
	\omega\propto p(u_{t+1},v_{t+1};\theta_{t+1})$ (proportional to the predictive density of $(v_{t+1},u_{t+1})$)
	The components of $\Theta=(\alpha,\phi,\delta)$ have been simulated at the end of the previous period. The resampled particles are denoted by $\tilde{\Theta}$, for example.
\end{itemize}  
\end{frame}
%%%%%%%%%%%%%%%%%%%%%%%%%%%
%%%%%%%%%%%%%%%%%%%%%%%%%%%

\begin{frame}{Sequential estimation for the SCAR model II}

	\begin{itemize}
	\item[3.]\textbf{Propagating sufficient statistics and learning $\Theta_c$:}
		\begin{itemize}
			\item[(3.1)]  Sample $\delta^2$ from $\text{IG}(\delta^2; b_0^{\star}/2,b_0^{\star}\delta_0^{2\star}/2)$, where
			$$
			b_0^{\star}=\tilde{b}_0+1\:\:\:\text{and}\:\:\:b_0^{\star}\delta_0^{2\star} = \tilde{b}_0\tilde{\delta}_0^2+\frac{(\tilde{m}_{\phi}\tilde{x}_{t}-(\tilde{x}_{t+1}-\tilde{\alpha}))^2}{1+\tilde{V}_{\phi}\tilde{x}_{t}^2}.\:\:
			$$
			\item[(3.2)] Sample $\phi$ from $\text{N}(\phi; m_{\phi}^{\star},V_{\phi}^{\star}\delta^2)$, where
			$$
			m_{\phi}^{\star}=\frac{\tilde{m}_{\phi}+\tilde{V}_{\phi}\tilde{x}_{t}(\tilde{x}_{t+1}-\tilde{\alpha})}{1+\tilde{V}_{\phi}\tilde{x}_{t}^2}\:\:\text{and}\:\:\:V_{\phi}^{\star} = \frac{\tilde{V}_{\phi}}{1+\tilde{V}_{\phi}\tilde{x}_{t}^2}.
			$$
			\item[(3.3)] Sample $\alpha$ from $\text{N}(\alpha; m_{\alpha}^{\star},V_{\alpha}^{\star})$, where
			$$
			m_{\alpha}^{\star} = \frac{\tilde{m}_{\alpha}\tau^2+\tilde{V}_{\alpha}(\tilde{x}_{t+1}-\phi\tilde{x}_{t})}{\delta^2+\tilde{V}_{\alpha}}\:\:\text{and}\:\:\:V_{\alpha}^{\star} = \frac{\delta^2 \tilde{V}_{\alpha}}{\delta^2+\tilde{V}_{\alpha}}.
			$$
		\end{itemize}
	\vspace{0.3cm}
	\item[4.]\textbf{Propagating sufficient statistics and learning $\Theta_v$:} as in 3. just for $\tau^{2(RV)}$, $\phi^{(RV)}$, and $
	\mu^{(RV)}$
	\end{itemize}  
\end{frame}
%%%%%%%%%%%%%%%%%%%%%%%%%%%
%%%%%%%%%%%%%%%%%%%%%%%%%%%

\section{Empirical Results}

\begin{frame}{Data and Set-up}
	\begin{itemize}
		\item Five US stocks from different industries: BA, IBM, JNJ, WMT, XOM 
		\item Daily observations from 01/01/2001 - 31/12/2015
		\item One-step ahead predictions
		\item Competing models:
		\begin{enumerate}
			\item Normal copula model
			\item Double Clayton copula model
			\item Double Gumbel copula model
			\item Zero mean model standardized with realized volatility
			\item Zero mean model standardized with stochastic volatility
			\item Static mean model (predictor: dividend yield) standardized with realized volatility
			\item Time-varying mean model (dividend yield) standardized with realized volatility
		\end{enumerate}
		\item (Preliminary) evaluation: Sequential predictive log-Bayes factors, LPTS
		
	\end{itemize}
\end{frame}
%%%%%%%%%%%%%%%%%%%%%%%%%%%
%%%%%%%%%%%%%%%%%%%%%%%%%%%

\begin{frame}{Evaluation - Average Log-Predictive Score (LPS)}
	\begin{itemize}
		\item Use average log-predictive score (LPS) and average log-predictive tail score (LPTS$_{\alpha}$) to compare the performance of the models:
		\begin{align*}
		\text{LPS}&=-\frac{1}{T} \sum\limits_{t=1}^T \log p(r_t|r^{t-1})\\
		\text{LPTS}_{\alpha}&=-\frac{1}{\sum\limits_{t=1}^T \mathbf{1}\{r_t>z_{\alpha}\}}\sum\limits_{t=1}^T\mathbf{1}\{r_t>z_{\alpha}\}\log p(r_t|r^{t-1}),
		\end{align*} 
		where $z_{\alpha}$ is the upper $100\cdot\alpha$ percentile of the empirical distribution of $r_t$. 
		
		\item Note that the LPTS$_{\alpha}$ is not a proper scoring rule, however, it can be very useful for understanding how the model performs in the tails.
	\end{itemize}
\end{frame}
%%%%%%%%%%%%%%%%%%%%%%%%%%%
%%%%%%%%%%%%%%%%%%%%%%%%%%%

\begin{frame}{Results for IBM I}
	\begin{figure}
		\centering
		\includegraphics[width=0.9\linewidth]{IBM_plot}
	\end{figure}
\end{frame}
%%%%%%%%%%%%%%%%%%%%%%%%%%%
%%%%%%%%%%%%%%%%%%%%%%%%%%%

\begin{frame}{Results for IBM II}
	\begin{figure}
		\centering
		\includegraphics[width=1\linewidth]{IBM1}
	\end{figure}
\end{frame}
%%%%%%%%%%%%%%%%%%%%%%%%%%%
%%%%%%%%%%%%%%%%%%%%%%%%%%%

\begin{frame}{Results for IBM III}
	\begin{figure}
		\centering
		\includegraphics[width=1\linewidth]{IBM2}
	\end{figure}
\end{frame}
%%%%%%%%%%%%%%%%%%%%%%%%%%%
%%%%%%%%%%%%%%%%%%%%%%%%%%%

\begin{frame}{Results for IBM IV}
	\begin{figure}
		\centering
		\includegraphics[width=1\linewidth]{IBM4}
	\end{figure}
\end{frame}
%%%%%%%%%%%%%%%%%%%%%%%%%%%
%%%%%%%%%%%%%%%%%%%%%%%%%%%


\begin{frame}{Results for IBM V}
	\begin{figure}
		\centering
		\includegraphics[width=1\linewidth]{IBM3}
	\end{figure}
\end{frame}
%%%%%%%%%%%%%%%%%%%%%%%%%%%
%%%%%%%%%%%%%%%%%%%%%%%%%%%

\begin{frame}{Results for IBM VI}
	\begin{figure}
		\centering
		\includegraphics[width=1\linewidth]{IBM_LPTS01}
	\end{figure}
\end{frame}
%%%%%%%%%%%%%%%%%%%%%%%%%%%
%%%%%%%%%%%%%%%%%%%%%%%%%%%

\begin{frame}{Results for IBM VII}
	\begin{figure}
		\centering
		\includegraphics[width=1\linewidth]{IBM_LPTS05}
	\end{figure}
\end{frame}
%
%\begin{frame}{Summary}
%\begin{enumerate}
%	\item All copula based models outperform the linear benchmark models at all times. This holds in terms of Bayes factors, but also in terms of log-predictive tail scores.
%	\item The Clayton copula dominates the Gausian and Gumbel copula. It accounts especially for lower-tail dependence. This contributes to the fact that financial returns are often left-skewed.
%	\item Using realized volatilities to standardize returns instead of stochastic volatilities is beneficial for all assets.
%	\item Models 6 and 7 also include exogenous variable: dividend yield. However, we do not find sufficient evidence that dividend yield is beneficial for return prediction, neither with static, nor with time-varying coefficient. This results contradicts the findings in \cite{Johannes2014}, however, this might be due to the fact that they used monthly data, where dividend yield is much more important return predictor as compared to daily data.	
%	
%\end{enumerate}
%
%\end{frame}
%%%%%%%%%%%%%%%%%%%%%%%%%%%%
%%%%%%%%%%%%%%%%%%%%%%%%%%%%
%%%%%%%%%%%%%%%%%%%%%%%%%%%
%%%%%%%%%%%%%%%%%%%%%%%%%%%
%
%\begin{frame}{Results for JPM I}
%	\begin{figure}
%		\centering
%		\includegraphics[width=0.9\linewidth]{JPM_plot}
%	\end{figure}
%\end{frame}
%%%%%%%%%%%%%%%%%%%%%%%%%%%%
%%%%%%%%%%%%%%%%%%%%%%%%%%%%
%
%\begin{frame}{Results for JPM II}
%	\begin{figure}
%		\centering
%		\includegraphics[width=1\linewidth]{JPM1}
%	\end{figure}
%\end{frame}
%%%%%%%%%%%%%%%%%%%%%%%%%%%%
%%%%%%%%%%%%%%%%%%%%%%%%%%%%
%
%\begin{frame}{Results for JPM III}
%	\begin{figure}
%		\centering
%		\includegraphics[width=1\linewidth]{JPM_LPTS01}
%	\end{figure}
%\end{frame}
%%%%%%%%%%%%%%%%%%%%%%%%%%%%
%%%%%%%%%%%%%%%%%%%%%%%%%%%%
%
%\begin{frame}{Results for JPM IV}
%	\begin{figure}
%		\centering
%		\includegraphics[width=1\linewidth]{JPM_LPTS05}
%	\end{figure}
%\end{frame}
%%%%%%%%%%%%%%%%%%%%%%%%%%%%
%%%%%%%%%%%%%%%%%%%%%%%%%%%%

\section{Conclusions}
\begin{frame}{Conclusions and Outlook}
	\begin{itemize}
		\item We model a time-varying stochastic copula to model time-varying dependence between a single financial asset and its past standardized by its realized volatility
		\item Model seems to outperform common benchmark models in terms of Bayes factor and log-predictive tail scores
		\item We do not find sufficient evidence that dividend yield is beneficial for return prediction, neither with static, nor with time-varying coefficient. %This results contradicts the findings in \cite{Johannes2014}, however, this might be due to the fact that they used monthly data	
	\end{itemize}
\pause
	\textbf{Outlook}
	\begin{itemize}
		\item HAR model for forecasting RV for a longer horizons
		\item Applications: Value-at-Risk 
		\item Extension: Multivariate copula modeling of marginal predictive distributions $\to$ portfolio application
	\end{itemize}
\end{frame}
%%%%%%%%%%%%%%%%%%%%%%%%%%%
%%%%%%%%%%%%%%%%%%%%%%%%%%%

\begin{frame}{References}
	\fontsize{5.5}{1}\selectfont{
		\setlength{\bibsep}{1\baselineskip}
		\bibliographystyle{ecta}
		\bibliography{./refs}}
\end{frame}
%%%%%%%%%%%%%%%%%%%%%%%%%%%
%%%%%%%%%%%%%%%%%%%%%%%%%%%

\appendix
\newcounter{finalframe}
\setcounter{finalframe}{\value{framenumber}}


\begin{frame}[plain]{Standardized returns by realized volatilities}\label{sampling}
	\begin{figure}
		\centering
		\includegraphics[width=0.8\linewidth]{rv}
	\end{figure}

\hyperlink{rv}{\beamergotobutton{Go back}}
\end{frame}
%%%%%%%%%%%%%%%%%%%%%%%%%%%
%%%%%%%%%%%%%%%%%%%%%%%%%%%
\setcounter{framenumber}{\value{finalframe}}

\end{document}